\documentclass[]{article}
\usepackage{lmodern}
\usepackage{amssymb,amsmath}
\usepackage{ifxetex,ifluatex}
\usepackage{fixltx2e} % provides \textsubscript
\ifnum 0\ifxetex 1\fi\ifluatex 1\fi=0 % if pdftex
  \usepackage[T1]{fontenc}
  \usepackage[utf8]{inputenc}
\else % if luatex or xelatex
  \ifxetex
    \usepackage{mathspec}
  \else
    \usepackage{fontspec}
  \fi
  \defaultfontfeatures{Ligatures=TeX,Scale=MatchLowercase}
\fi
% use upquote if available, for straight quotes in verbatim environments
\IfFileExists{upquote.sty}{\usepackage{upquote}}{}
% use microtype if available
\IfFileExists{microtype.sty}{%
\usepackage{microtype}
\UseMicrotypeSet[protrusion]{basicmath} % disable protrusion for tt fonts
}{}
\usepackage[margin=1in]{geometry}
\usepackage{hyperref}
\hypersetup{unicode=true,
            pdftitle={Arctic Sea Ice},
            pdfauthor={Kyle Oliveira, Arianna Torello, Meg VanHorn},
            pdfborder={0 0 0},
            breaklinks=true}
\urlstyle{same}  % don't use monospace font for urls
\usepackage{color}
\usepackage{fancyvrb}
\newcommand{\VerbBar}{|}
\newcommand{\VERB}{\Verb[commandchars=\\\{\}]}
\DefineVerbatimEnvironment{Highlighting}{Verbatim}{commandchars=\\\{\}}
% Add ',fontsize=\small' for more characters per line
\usepackage{framed}
\definecolor{shadecolor}{RGB}{248,248,248}
\newenvironment{Shaded}{\begin{snugshade}}{\end{snugshade}}
\newcommand{\KeywordTok}[1]{\textcolor[rgb]{0.13,0.29,0.53}{\textbf{#1}}}
\newcommand{\DataTypeTok}[1]{\textcolor[rgb]{0.13,0.29,0.53}{#1}}
\newcommand{\DecValTok}[1]{\textcolor[rgb]{0.00,0.00,0.81}{#1}}
\newcommand{\BaseNTok}[1]{\textcolor[rgb]{0.00,0.00,0.81}{#1}}
\newcommand{\FloatTok}[1]{\textcolor[rgb]{0.00,0.00,0.81}{#1}}
\newcommand{\ConstantTok}[1]{\textcolor[rgb]{0.00,0.00,0.00}{#1}}
\newcommand{\CharTok}[1]{\textcolor[rgb]{0.31,0.60,0.02}{#1}}
\newcommand{\SpecialCharTok}[1]{\textcolor[rgb]{0.00,0.00,0.00}{#1}}
\newcommand{\StringTok}[1]{\textcolor[rgb]{0.31,0.60,0.02}{#1}}
\newcommand{\VerbatimStringTok}[1]{\textcolor[rgb]{0.31,0.60,0.02}{#1}}
\newcommand{\SpecialStringTok}[1]{\textcolor[rgb]{0.31,0.60,0.02}{#1}}
\newcommand{\ImportTok}[1]{#1}
\newcommand{\CommentTok}[1]{\textcolor[rgb]{0.56,0.35,0.01}{\textit{#1}}}
\newcommand{\DocumentationTok}[1]{\textcolor[rgb]{0.56,0.35,0.01}{\textbf{\textit{#1}}}}
\newcommand{\AnnotationTok}[1]{\textcolor[rgb]{0.56,0.35,0.01}{\textbf{\textit{#1}}}}
\newcommand{\CommentVarTok}[1]{\textcolor[rgb]{0.56,0.35,0.01}{\textbf{\textit{#1}}}}
\newcommand{\OtherTok}[1]{\textcolor[rgb]{0.56,0.35,0.01}{#1}}
\newcommand{\FunctionTok}[1]{\textcolor[rgb]{0.00,0.00,0.00}{#1}}
\newcommand{\VariableTok}[1]{\textcolor[rgb]{0.00,0.00,0.00}{#1}}
\newcommand{\ControlFlowTok}[1]{\textcolor[rgb]{0.13,0.29,0.53}{\textbf{#1}}}
\newcommand{\OperatorTok}[1]{\textcolor[rgb]{0.81,0.36,0.00}{\textbf{#1}}}
\newcommand{\BuiltInTok}[1]{#1}
\newcommand{\ExtensionTok}[1]{#1}
\newcommand{\PreprocessorTok}[1]{\textcolor[rgb]{0.56,0.35,0.01}{\textit{#1}}}
\newcommand{\AttributeTok}[1]{\textcolor[rgb]{0.77,0.63,0.00}{#1}}
\newcommand{\RegionMarkerTok}[1]{#1}
\newcommand{\InformationTok}[1]{\textcolor[rgb]{0.56,0.35,0.01}{\textbf{\textit{#1}}}}
\newcommand{\WarningTok}[1]{\textcolor[rgb]{0.56,0.35,0.01}{\textbf{\textit{#1}}}}
\newcommand{\AlertTok}[1]{\textcolor[rgb]{0.94,0.16,0.16}{#1}}
\newcommand{\ErrorTok}[1]{\textcolor[rgb]{0.64,0.00,0.00}{\textbf{#1}}}
\newcommand{\NormalTok}[1]{#1}
\usepackage{graphicx,grffile}
\makeatletter
\def\maxwidth{\ifdim\Gin@nat@width>\linewidth\linewidth\else\Gin@nat@width\fi}
\def\maxheight{\ifdim\Gin@nat@height>\textheight\textheight\else\Gin@nat@height\fi}
\makeatother
% Scale images if necessary, so that they will not overflow the page
% margins by default, and it is still possible to overwrite the defaults
% using explicit options in \includegraphics[width, height, ...]{}
\setkeys{Gin}{width=\maxwidth,height=\maxheight,keepaspectratio}
\IfFileExists{parskip.sty}{%
\usepackage{parskip}
}{% else
\setlength{\parindent}{0pt}
\setlength{\parskip}{6pt plus 2pt minus 1pt}
}
\setlength{\emergencystretch}{3em}  % prevent overfull lines
\providecommand{\tightlist}{%
  \setlength{\itemsep}{0pt}\setlength{\parskip}{0pt}}
\setcounter{secnumdepth}{0}
% Redefines (sub)paragraphs to behave more like sections
\ifx\paragraph\undefined\else
\let\oldparagraph\paragraph
\renewcommand{\paragraph}[1]{\oldparagraph{#1}\mbox{}}
\fi
\ifx\subparagraph\undefined\else
\let\oldsubparagraph\subparagraph
\renewcommand{\subparagraph}[1]{\oldsubparagraph{#1}\mbox{}}
\fi

%%% Use protect on footnotes to avoid problems with footnotes in titles
\let\rmarkdownfootnote\footnote%
\def\footnote{\protect\rmarkdownfootnote}

%%% Change title format to be more compact
\usepackage{titling}

% Create subtitle command for use in maketitle
\newcommand{\subtitle}[1]{
  \posttitle{
    \begin{center}\large#1\end{center}
    }
}

\setlength{\droptitle}{-2em}

  \title{Arctic Sea Ice}
    \pretitle{\vspace{\droptitle}\centering\huge}
  \posttitle{\par}
    \author{Kyle Oliveira, Arianna Torello, Meg VanHorn}
    \preauthor{\centering\large\emph}
  \postauthor{\par}
      \predate{\centering\large\emph}
  \postdate{\par}
    \date{1/30/2020}


\begin{document}
\maketitle

\subsection{Sea Ice}\label{sea-ice}

This is an R Markdown document to help walk through the process of our
code that examines the Arctic sea ice area through the period of
November 1978 to December 2019.

First, we will need to import our data.

\begin{Shaded}
\begin{Highlighting}[]
\NormalTok{the_file<-}\StringTok{"~/Desktop/8thSemester/GEOG456/data/SeaIceMaster.csv"}
\NormalTok{df<-}\KeywordTok{read.csv}\NormalTok{(the_file)}
\end{Highlighting}
\end{Shaded}

Let's create a better column for the date by combining together the
month and year columns.

\begin{Shaded}
\begin{Highlighting}[]
\NormalTok{df}\OperatorTok{$}\NormalTok{date <-}\StringTok{ }\KeywordTok{with}\NormalTok{(df, }\KeywordTok{sprintf}\NormalTok{(}\StringTok{"%d-%02d"}\NormalTok{, year, mo))}
\end{Highlighting}
\end{Shaded}

We will need a few packages to complete this work so let us install
those.

\begin{Shaded}
\begin{Highlighting}[]
\KeywordTok{library}\NormalTok{(tidyr)}
\KeywordTok{library}\NormalTok{(dplyr)}
\end{Highlighting}
\end{Shaded}

\begin{verbatim}
## 
## Attaching package: 'dplyr'
\end{verbatim}

\begin{verbatim}
## The following objects are masked from 'package:stats':
## 
##     filter, lag
\end{verbatim}

\begin{verbatim}
## The following objects are masked from 'package:base':
## 
##     intersect, setdiff, setequal, union
\end{verbatim}

\begin{Shaded}
\begin{Highlighting}[]
\KeywordTok{library}\NormalTok{(lubridate)}
\end{Highlighting}
\end{Shaded}

\begin{verbatim}
## 
## Attaching package: 'lubridate'
\end{verbatim}

\begin{verbatim}
## The following object is masked from 'package:base':
## 
##     date
\end{verbatim}

\begin{Shaded}
\begin{Highlighting}[]
\KeywordTok{library}\NormalTok{(ggplot2)}
\KeywordTok{library}\NormalTok{(RColorBrewer)}
\end{Highlighting}
\end{Shaded}

Let's also find the average area of the sea ice in our data set. This
will come in use in our analysis later.

\begin{Shaded}
\begin{Highlighting}[]
\NormalTok{avg_area <-}\StringTok{ }\KeywordTok{mean}\NormalTok{(df}\OperatorTok{$}\NormalTok{area)}
\end{Highlighting}
\end{Shaded}

And let's make the month column a character value instead of a numeric
value.

\begin{Shaded}
\begin{Highlighting}[]
\NormalTok{mymonths <-}\StringTok{ }\KeywordTok{c}\NormalTok{(}\StringTok{"Jan"}\NormalTok{,}\StringTok{"Feb"}\NormalTok{,}\StringTok{"Mar"}\NormalTok{,}
              \StringTok{"Apr"}\NormalTok{,}\StringTok{"May"}\NormalTok{,}\StringTok{"Jun"}\NormalTok{,}
              \StringTok{"Jul"}\NormalTok{,}\StringTok{"Aug"}\NormalTok{,}\StringTok{"Sep"}\NormalTok{,}
              \StringTok{"Oct"}\NormalTok{,}\StringTok{"Nov"}\NormalTok{,}\StringTok{"Dec"}\NormalTok{)}
\NormalTok{df}\OperatorTok{$}\NormalTok{month <-}\StringTok{ }\NormalTok{mymonths[ df}\OperatorTok{$}\NormalTok{mo ]}
\end{Highlighting}
\end{Shaded}

Great! Now that we have all of that settled, lets begin to make our
figures!

For our first figure lets just get a feel for what we're working with.
Let's create a dot plot that shows the extent of the ice. But we want to
include area as well, so lets change the size of the dots to reflect how
area has changed as well. We will want to see how these two features
have changed over time so lets put the year (our time value) on the
x-axis and make the color of our dots equal to year as well for an
aesthetically pleasing gradient.

\begin{Shaded}
\begin{Highlighting}[]
\KeywordTok{ggplot}\NormalTok{(df, }\KeywordTok{aes}\NormalTok{(}\DataTypeTok{x =}\NormalTok{ year, }\DataTypeTok{y =}\NormalTok{ extent, }\DataTypeTok{size =}\NormalTok{ area, }\DataTypeTok{color =}\NormalTok{ year)) }\OperatorTok{+}
\StringTok{  }\KeywordTok{geom_point}\NormalTok{()}
\end{Highlighting}
\end{Shaded}

\includegraphics{SeaIceMarkdown_files/figure-latex/unnamed-chunk-6-1.pdf}

That looks great. Let's call this figure our Figure 1. That way when we
call fig1 it will produce this picture.

\begin{Shaded}
\begin{Highlighting}[]
\NormalTok{fig1<-}\KeywordTok{ggplot}\NormalTok{(df, }\KeywordTok{aes}\NormalTok{(}\DataTypeTok{x =}\NormalTok{ year, }\DataTypeTok{y =}\NormalTok{ extent, }\DataTypeTok{size =}\NormalTok{ area, }\DataTypeTok{color =}\NormalTok{ year)) }\OperatorTok{+}
\StringTok{  }\KeywordTok{geom_point}\NormalTok{()}
\end{Highlighting}
\end{Shaded}

It is hard to tell the difference between extent and area in our first
figure so lets create a line plot that shows how the two differ. Our red
line will be the area, and the blue line with grey background will be
the extent of the Arctic sea ice. We'll rename the y-axis to be more
descriptive, and change the theme so the standard deviation of the
extent doesn't blend with the background. We won't need a legend since
the description of colors is in the y-axis.

\begin{Shaded}
\begin{Highlighting}[]
\KeywordTok{ggplot}\NormalTok{(df)}\OperatorTok{+}
\StringTok{  }\KeywordTok{geom_smooth}\NormalTok{(}\KeywordTok{aes}\NormalTok{(}\DataTypeTok{x=}\NormalTok{year, }\DataTypeTok{y=}\NormalTok{extent))}\OperatorTok{+}
\StringTok{  }\KeywordTok{geom_smooth}\NormalTok{(}\KeywordTok{aes}\NormalTok{(}\DataTypeTok{x=}\NormalTok{year, }\DataTypeTok{y=}\NormalTok{area, }\DataTypeTok{color=}\StringTok{"red"}\NormalTok{ ,}\DataTypeTok{fill=}\StringTok{"red"}\NormalTok{))}\OperatorTok{+}
\StringTok{  }\KeywordTok{expand_limits}\NormalTok{(}\DataTypeTok{y=}\DecValTok{0}\NormalTok{)}\OperatorTok{+}
\StringTok{  }\KeywordTok{theme_bw}\NormalTok{() }\OperatorTok{+}
\StringTok{  }\KeywordTok{labs}\NormalTok{(}\DataTypeTok{x =} \StringTok{"Year"}\NormalTok{,}\DataTypeTok{y=}\StringTok{'Extent(blue)/ Area(red) ('}\OperatorTok{~}\NormalTok{km}\OperatorTok{^}\DecValTok{2}\OperatorTok{~}\NormalTok{x10}\OperatorTok{^}\DecValTok{6}\OperatorTok{*}\StringTok{')'}\NormalTok{)}\OperatorTok{+}
\StringTok{  }\KeywordTok{theme}\NormalTok{(}
    \DataTypeTok{panel.grid.major =} \KeywordTok{element_blank}\NormalTok{(),}
    \DataTypeTok{panel.grid.minor =} \KeywordTok{element_blank}\NormalTok{(),}
       \DataTypeTok{legend.position =} \StringTok{"none"}
\NormalTok{  )}
\end{Highlighting}
\end{Shaded}

\begin{verbatim}
## `geom_smooth()` using method = 'loess' and formula 'y ~ x'
## `geom_smooth()` using method = 'loess' and formula 'y ~ x'
\end{verbatim}

\includegraphics{SeaIceMarkdown_files/figure-latex/unnamed-chunk-8-1.pdf}

Looks great! Lets name this figure 2 and have it saved as such so we can
call on it when needed.

\begin{Shaded}
\begin{Highlighting}[]
\NormalTok{fig2<-}\KeywordTok{ggplot}\NormalTok{(df)}\OperatorTok{+}
\StringTok{  }\KeywordTok{geom_smooth}\NormalTok{(}\KeywordTok{aes}\NormalTok{(}\DataTypeTok{x=}\NormalTok{year, }\DataTypeTok{y=}\NormalTok{extent))}\OperatorTok{+}
\StringTok{  }\KeywordTok{geom_smooth}\NormalTok{(}\KeywordTok{aes}\NormalTok{(}\DataTypeTok{x=}\NormalTok{year, }\DataTypeTok{y=}\NormalTok{area, }\DataTypeTok{color=}\StringTok{"red"}\NormalTok{ ,}\DataTypeTok{fill=}\StringTok{"red"}\NormalTok{))}\OperatorTok{+}
\StringTok{  }\KeywordTok{expand_limits}\NormalTok{(}\DataTypeTok{y=}\DecValTok{0}\NormalTok{)}\OperatorTok{+}
\StringTok{  }\KeywordTok{theme_bw}\NormalTok{() }\OperatorTok{+}
\StringTok{  }\KeywordTok{labs}\NormalTok{(}\DataTypeTok{x =} \StringTok{"Year"}\NormalTok{,}\DataTypeTok{y=}\StringTok{'Extent (blue) / Area (red) ('}\OperatorTok{~}\NormalTok{km}\OperatorTok{^}\DecValTok{2}\OperatorTok{~}\NormalTok{x10}\OperatorTok{^}\DecValTok{6}\OperatorTok{*}\StringTok{')'}\NormalTok{)}\OperatorTok{+}
\StringTok{  }\KeywordTok{theme}\NormalTok{(}
    \DataTypeTok{panel.grid.major =} \KeywordTok{element_blank}\NormalTok{(),}
    \DataTypeTok{panel.grid.minor =} \KeywordTok{element_blank}\NormalTok{(),}
    \DataTypeTok{legend.position =} \StringTok{"none"}
\NormalTok{  )}
\end{Highlighting}
\end{Shaded}

So now that we've seen how area and extent differ, let's work with just
area, because it gives a better depiciton of how the ice ice changing as
a whole and not just how far the ice extends in the Arctic.

Since we are working with just area, let's see how it changes temporally
(throughout the year) and also how the area has changed over the years.
To do this we'll create a line plot that shows the temporal changes, but
each line will be an individual year. To do this, we will need to change
our year variable into a factor, so we can use it to be individual
inputs.

\begin{Shaded}
\begin{Highlighting}[]
\NormalTok{df}\OperatorTok{$}\NormalTok{yearf <-}\StringTok{ }\KeywordTok{as.factor}\NormalTok{(df}\OperatorTok{$}\NormalTok{year)}
\end{Highlighting}
\end{Shaded}

Now that that's done, we can make our figure. Don't forget to add units
to the y-axis so the viewer knows what we're working with.

\begin{Shaded}
\begin{Highlighting}[]
\KeywordTok{ggplot}\NormalTok{(df, }\KeywordTok{aes}\NormalTok{(}\DataTypeTok{x=}\NormalTok{mo, }\DataTypeTok{y=}\NormalTok{area, }\DataTypeTok{color=}\NormalTok{yearf))}\OperatorTok{+}
\StringTok{  }\KeywordTok{expand_limits}\NormalTok{(}\DataTypeTok{y=}\DecValTok{0}\NormalTok{)}\OperatorTok{+}
\StringTok{  }\KeywordTok{coord_cartesian}\NormalTok{(}\DataTypeTok{xlim =} \KeywordTok{c}\NormalTok{(}\DecValTok{1}\NormalTok{,}\DecValTok{12}\NormalTok{))}\OperatorTok{+}
\StringTok{  }\KeywordTok{labs}\NormalTok{(}\DataTypeTok{x =} \StringTok{"Month"}\NormalTok{, }\DataTypeTok{y=}\StringTok{'Area ('}\OperatorTok{~}\NormalTok{km}\OperatorTok{^}\DecValTok{2}\OperatorTok{~}\NormalTok{x10}\OperatorTok{^}\DecValTok{6}\OperatorTok{*}\StringTok{')'}\NormalTok{, }\DataTypeTok{color=}\StringTok{"Year"}\NormalTok{)}\OperatorTok{+}
\StringTok{  }\KeywordTok{theme_classic}\NormalTok{()}\OperatorTok{+}
\StringTok{  }\KeywordTok{geom_line}\NormalTok{()}
\end{Highlighting}
\end{Shaded}

\includegraphics{SeaIceMarkdown_files/figure-latex/unnamed-chunk-11-1.pdf}

Wow! You can really see how the ice area reaches a low point in the late
summer and early fall months. You can also really see how in recent
years the ice area has been much less than in earlier years in the data
set.

Let's save this figure as figure 3 for later.

\begin{Shaded}
\begin{Highlighting}[]
\NormalTok{fig3<-}\KeywordTok{ggplot}\NormalTok{(df, }\KeywordTok{aes}\NormalTok{(}\DataTypeTok{x=}\NormalTok{mo, }\DataTypeTok{y=}\NormalTok{area, }\DataTypeTok{color=}\NormalTok{year))}\OperatorTok{+}
\StringTok{  }\KeywordTok{expand_limits}\NormalTok{(}\DataTypeTok{y=}\DecValTok{0}\NormalTok{)}\OperatorTok{+}
\StringTok{  }\KeywordTok{coord_cartesian}\NormalTok{(}\DataTypeTok{xlim =} \KeywordTok{c}\NormalTok{(}\DecValTok{1}\NormalTok{,}\DecValTok{12}\NormalTok{))}\OperatorTok{+}
\StringTok{  }\KeywordTok{labs}\NormalTok{(}\DataTypeTok{x =} \StringTok{"Month"}\NormalTok{, }\DataTypeTok{y=}\StringTok{'Area ('}\OperatorTok{~}\NormalTok{km}\OperatorTok{^}\DecValTok{2}\OperatorTok{~}\NormalTok{x10}\OperatorTok{^}\DecValTok{6}\OperatorTok{*}\StringTok{')'}\NormalTok{, }\DataTypeTok{color=}\StringTok{"Year"}\NormalTok{)}\OperatorTok{+}
\StringTok{  }\KeywordTok{theme_classic}\NormalTok{()}\OperatorTok{+}
\StringTok{  }\KeywordTok{geom_line}\NormalTok{()}
\end{Highlighting}
\end{Shaded}

Let's play around a bit and make an aesthetically pleasing graph that
really makes the viewer interested and want to ask more. A good way to
do this would be to create a circular graph that can show how the ice
has grown and shrunk with respect to its average ice area (which we
calculated earlier). So if we create a histogram and wrap that around a
central point, we can end up with this type of graph. The graph will
begin with November 1978 being at the very top of the graph and will
circle all the way around so December of 2019 will be the last entry.
Think of this graph as a clock with November of 1978 as 12:00 and with
December of 2019 as 11:59.

\begin{Shaded}
\begin{Highlighting}[]
\KeywordTok{ggplot}\NormalTok{(df)}\OperatorTok{+}
\StringTok{  }\KeywordTok{geom_bar}\NormalTok{(}\KeywordTok{aes}\NormalTok{(}\DataTypeTok{x=}\NormalTok{date, }\DataTypeTok{y=}\NormalTok{area}\OperatorTok{-}\NormalTok{avg_area, }\DataTypeTok{fill =}\NormalTok{ area}\OperatorTok{-}\NormalTok{avg_area), }\DataTypeTok{stat =} \StringTok{"identity"}\NormalTok{)}\OperatorTok{+}
\StringTok{  }\KeywordTok{scale_fill_gradientn}\NormalTok{(}\DataTypeTok{colours =} \KeywordTok{colorRampPalette}\NormalTok{(}\KeywordTok{brewer.pal}\NormalTok{(}\DecValTok{11}\NormalTok{, }\StringTok{"RdBu"}\NormalTok{))(}\DecValTok{11}\NormalTok{))}\OperatorTok{+}
\StringTok{  }\KeywordTok{theme_minimal}\NormalTok{() }\OperatorTok{+}\StringTok{ }
\StringTok{  }\KeywordTok{theme}\NormalTok{(}
    \DataTypeTok{axis.text.x =} \KeywordTok{element_blank}\NormalTok{(),}
    \DataTypeTok{axis.title.x =} \KeywordTok{element_blank}\NormalTok{(),}
    \DataTypeTok{panel.grid.major =} \KeywordTok{element_blank}\NormalTok{(),}
    \DataTypeTok{panel.grid.minor =} \KeywordTok{element_blank}\NormalTok{(),}
\NormalTok{  )}\OperatorTok{+}
\StringTok{  }\KeywordTok{scale_y_discrete}\NormalTok{(}\DataTypeTok{name=}\StringTok{'Area ('}\OperatorTok{~}\NormalTok{km}\OperatorTok{^}\DecValTok{2}\OperatorTok{~}\NormalTok{x10}\OperatorTok{^}\DecValTok{6}\OperatorTok{*}\StringTok{')'}\NormalTok{, }\DataTypeTok{limits=}\KeywordTok{c}\NormalTok{(}\OperatorTok{-}\DecValTok{7}\NormalTok{, }\DecValTok{0}\NormalTok{, }\DecValTok{5}\NormalTok{))}\OperatorTok{+}
\StringTok{  }\KeywordTok{labs}\NormalTok{(}\DataTypeTok{title =} \StringTok{"Area Difference of Arctic Sea Ice over time (1978-2019)"}\NormalTok{, }\DataTypeTok{fill =} \StringTok{'Area ('}\OperatorTok{~}\NormalTok{km}\OperatorTok{^}\DecValTok{2}\OperatorTok{~}\NormalTok{x10}\OperatorTok{^}\DecValTok{6}\OperatorTok{*}\StringTok{')'}\NormalTok{)}\OperatorTok{+}
\StringTok{  }\KeywordTok{coord_polar}\NormalTok{(}\DataTypeTok{theta =} \StringTok{"x"}\NormalTok{)}
\end{Highlighting}
\end{Shaded}

\includegraphics{SeaIceMarkdown_files/figure-latex/unnamed-chunk-13-1.pdf}

This is really cool! In this graph we can see how the ice switches back
and forth between negative and positive values for ice area throughout
the years. But what really stands out is how much shorter positive
values have gotten in recent years and also how much deeper negative
values go in recent years as well. We see a much deeper red from about
9:00 onward to 11:59. Looking at this graph we can see the fluctuation
of ice area really well, and also the yearly trends in decreased ice
area.

Let's save this figure as figure 4.

\begin{Shaded}
\begin{Highlighting}[]
\NormalTok{fig4<-}\KeywordTok{ggplot}\NormalTok{(df)}\OperatorTok{+}
\StringTok{  }\KeywordTok{geom_bar}\NormalTok{(}\KeywordTok{aes}\NormalTok{(}\DataTypeTok{x=}\NormalTok{date, }\DataTypeTok{y=}\NormalTok{area}\OperatorTok{-}\NormalTok{avg_area, }\DataTypeTok{fill =}\NormalTok{ area}\OperatorTok{-}\NormalTok{avg_area), }\DataTypeTok{stat =} \StringTok{"identity"}\NormalTok{)}\OperatorTok{+}
\StringTok{  }\KeywordTok{scale_fill_gradientn}\NormalTok{(}\DataTypeTok{colours =} \KeywordTok{colorRampPalette}\NormalTok{(}\KeywordTok{brewer.pal}\NormalTok{(}\DecValTok{11}\NormalTok{, }\StringTok{"RdBu"}\NormalTok{))(}\DecValTok{11}\NormalTok{))}\OperatorTok{+}
\StringTok{  }\KeywordTok{theme_minimal}\NormalTok{() }\OperatorTok{+}\StringTok{ }
\StringTok{  }\KeywordTok{theme}\NormalTok{(}
    \DataTypeTok{axis.text.x =} \KeywordTok{element_blank}\NormalTok{(),}
    \DataTypeTok{axis.title.x =} \KeywordTok{element_blank}\NormalTok{(),}
    \DataTypeTok{panel.grid.major =} \KeywordTok{element_blank}\NormalTok{(),}
    \DataTypeTok{panel.grid.minor =} \KeywordTok{element_blank}\NormalTok{(),}
\NormalTok{  )}\OperatorTok{+}
\StringTok{  }\KeywordTok{scale_y_discrete}\NormalTok{(}\DataTypeTok{name=}\StringTok{'Area ('}\OperatorTok{~}\NormalTok{km}\OperatorTok{^}\DecValTok{2}\OperatorTok{~}\NormalTok{x10}\OperatorTok{^}\DecValTok{6}\OperatorTok{*}\StringTok{')'}\NormalTok{, }\DataTypeTok{limits=}\KeywordTok{c}\NormalTok{(}\OperatorTok{-}\DecValTok{7}\NormalTok{, }\DecValTok{0}\NormalTok{, }\DecValTok{5}\NormalTok{))}\OperatorTok{+}
\StringTok{  }\KeywordTok{labs}\NormalTok{(}\DataTypeTok{title =} \StringTok{"Area Difference of Arctic Sea Ice over time (1978-2019)"}\NormalTok{, }\DataTypeTok{fill =} \StringTok{'Area ('}\OperatorTok{~}\NormalTok{km}\OperatorTok{^}\DecValTok{2}\OperatorTok{~}\NormalTok{x10}\OperatorTok{^}\DecValTok{6}\OperatorTok{*}\StringTok{')'}\NormalTok{)}\OperatorTok{+}
\StringTok{  }\KeywordTok{coord_polar}\NormalTok{(}\DataTypeTok{theta =} \StringTok{"x"}\NormalTok{)}
\end{Highlighting}
\end{Shaded}

Lets see all four of our figures in one place to end this walk-through.

\begin{Shaded}
\begin{Highlighting}[]
\NormalTok{fig1}
\end{Highlighting}
\end{Shaded}

\includegraphics{SeaIceMarkdown_files/figure-latex/unnamed-chunk-15-1.pdf}

\begin{Shaded}
\begin{Highlighting}[]
\NormalTok{fig2}
\end{Highlighting}
\end{Shaded}

\begin{verbatim}
## `geom_smooth()` using method = 'loess' and formula 'y ~ x'
## `geom_smooth()` using method = 'loess' and formula 'y ~ x'
\end{verbatim}

\includegraphics{SeaIceMarkdown_files/figure-latex/unnamed-chunk-15-2.pdf}

\begin{Shaded}
\begin{Highlighting}[]
\NormalTok{fig3}
\end{Highlighting}
\end{Shaded}

\includegraphics{SeaIceMarkdown_files/figure-latex/unnamed-chunk-15-3.pdf}

\begin{Shaded}
\begin{Highlighting}[]
\NormalTok{fig4}
\end{Highlighting}
\end{Shaded}

\includegraphics{SeaIceMarkdown_files/figure-latex/unnamed-chunk-15-4.pdf}

These plots are unique and fun ways to visual changes in Sea Ice extent
and area over time. We hope you've enjoyed coding with us, and have
learned a bit more about plotting in RStudio!

Acknowledgments: Thanks to the National Snow and Ice Data Center for
collecting the Sea Ice Index data, which we utilized in this program.


\end{document}
